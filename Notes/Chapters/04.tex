\chapter{The Itô Formula and the Martingale Representation Theorem}

\section{1-Dimensional Itô Processes}

Itô formula: Integral version of the Chain Rule

Idea: introduce \textbf{Itô processes (stochastic integrals)} as sums of a  $\mathrm{d}B_s-$ and a $\mathrm{d}s$-integral. This family of integrals is stable under smooth maps (what does this mean?)

\begin{definition}[1-dimensional Itô processes]\label{def:ito-process}
    $B_t$ 1-dimensional Brownian motion on $(\Omega, \mathfrak{F}, \mathbb{P})$. A 1-dimensional Itô process (or stochastic integral) is a stochastic process $X_t$ on $(\Omega, \mathfrak{F}, \mathbb{P})$ of the form 
    \begin{equation}\label{eq:1d-ito}
        X_t = X_0 + \int_0^t u(s, \omega) ~\mathrm{d}s + \int_0^t v(s, \omega) ~\mathrm{d}B_s
    \end{equation}
    where $v \in \mathfrak{W}_\mathfrak{H}$ and
    \begin{equation*}
        \mathbb{P} \left[ \int_0^t v(s,\omega)^2 ~\mathrm{d}s < \infty \text{ for all } t \geq 0 \right] = 1
    \end{equation*}
    and $u$ is $\mathfrak{H}_t$-adapted (as in the condition 2') and 
    \begin{equation*}
        \mathbb{P} \left[ \int_0^t |u(s,\omega)| ~\mathrm{d}s < \infty \text{ for all } t \geq 0 \right] = 1
    \end{equation*}
\end{definition}

If $X_t$ is an Itô process of this form, then the equation \eqref{eq:1d-ito} is written in the differential form
\begin{equation*}
    \mathrm{d}X_t = u \mathrm{d}t + v \mathrm{d}B_t
\end{equation*}

Hence, we can write
\begin{equation*}
    \mathrm{d}\left( \frac{1}{2} B_t^2 \right) = \frac{1}{2} \mathrm{d}t + B_t \mathrm{d}B_t
\end{equation*}

\begin{theorem}[The 1-dimensional Itô formula]
    $X_t$ be an Itô process given by 
    \begin{equation*}
        \mathrm{d}X_t = u \mathrm{d}t + v \mathrm{d}B_t
    \end{equation*}
    and $g(t,x) \in \mathfrak{C}^2([0, \infty) \times \textbf{R})$. Then 
    \[
        Y_t = g(t, X_t)
    \]
    is an Itô process, and
    \begin{equation}
        \mathrm{d}Y_t = \frac{\partial g}{\partial t}(t, X_t) \mathrm{d}t + \frac{\partial g}{\partial x}(t, X_t) \mathrm{d}X_t + \frac{1}{2}\frac{\partial^2 g}{\partial x^2}(t, X_t)(\mathrm{d}X_t)^2
    \end{equation}
    where $(\mathrm{d}X_t)^2$ is computed according to the rules
    \[
        \mathrm{d}t \cdot \mathrm{d}t = \mathrm{d}t \cdot \mathrm{d}B_t = \mathrm{d}B_t \cdot \mathrm{d}t = 0, \, \mathrm{d}B_t \cdot \mathrm{d}B_t = dt
    \]
\end{theorem}

\begin{proof}
    Notice that the claim is equivalent to the expression  
    \[
        g(t, X_t) = g(0, X_0) + \int_0^t \left( \frac{\partial g}{\partial s}(s, X_s) + u_s \frac{\partial g}{\partial x} (s, X_s) + \frac{1}{2} v_s^2 \frac{\partial^2 g}{\partial x^2} (s, X_s) \right) ~\mathrm{d}s + \int_0^t v_s \frac{\partial g}{\partial x} (s, X_s) ~\mathrm{d}B_s
    \]
    where $u_s = u(s, \omega)$ and $v_s = v(s, \omega)$. 

    We'll assume that $g$ and its derivatives above are bounded. In the general case, we can approximate $g$ by $\mathfrak{C}^2$ functions that converge uniformly to $g$.

    Then write the Taylor expansion with respect to the arguments (in this case, $t$ and $X_t$), taking the first-order terms for all arguments with zero quadratic variation (here, $t$) and up to the second order for every argument with non zero quadratic variation (here, $X_t$).

    Notice that the terms with crossed $\Delta t_j$ and $\Delta B_j$ go to zero by Itô isometry.
    
    Also notice that the terms with crossed $(\Delta B_i)^2 - \Delta t_i$ and $(\Delta B_j)^2 - \Delta t_j$, $i \neq j$, vanish because they are indepedent. For the similar terms with $i = j$, the sum converges to a $\mathrm{d}t$ integral, showing that $(\mathrm{d} B_t)^2 = \mathrm{d}t$. 

    Since the remainder goes to zero as $\Delta t_j \to 0$, we complete the proof.
\end{proof}

\begin{example}
    Consider again the following integral
    \begin{equation*}
        I = \int_0^t B_s ~\mathrm{d}B_s
    \end{equation*}

    Choose $X_t = B_t$ and $g(t,x) = \frac{1}{2}x^2$, then 
    \[
        Y_t = g(t,B_t) = \frac{1}{2}B_t^2
    \]

    Then, by Itô's formula,
    \begin{equation*}
        \begin{aligned}
            \mathrm{d}Y_t   &= \frac{\partial g}{\partial t} \mathrm{d}t + \frac{\partial g}{\partial x} \mathrm{d}B_t + \frac{1}{2}\frac{\partial^2 g}{\partial x^2} (\mathrm{d}B_t)^2 \\
                            &= B_t \mathrm{d}B_t + \frac{1}{2}(\mathrm{d}B_t)^2 \\
                            &= B_t \mathrm{d}B_t + \frac{1}{2} \mathrm{d}t
        \end{aligned}
    \end{equation*}
    I.e.,
    \[
        \mathrm{d}\left( \frac{1}{2} B_t^2 \right) = B_t \mathrm{d}B_t + \frac{1}{2} \mathrm{d}t
    \]

    Put it another way,
    \begin{equation*}
        \frac{1}{2} B_t^2 = \int_0^t B_s ~\mathrm{d}B_s + \frac{1}{2}t
    \end{equation*}
\end{example}

\begin{example}
    Now consider
    \[
        \int_0^t s ~\mathrm{d}B_s
    \]

    Put
    \[
        g(t,x) = tx \text{ and } Y_t = g(t,B_t) = tB_t
    \]

    By Itô's formula,
    \[
        \mathrm{d}Y_t = \frac{\partial g}{\partial t} \mathrm{d}t + \frac{\partial g}{\partial x} \mathrm{d}B_t + \frac{1}{2}\frac{\partial^2 g}{\partial x^2} (\mathrm{d}B_t)^2 
    \]
    i.e.
    \[
        \mathrm{d}(tB_t) = \mathrm{d}Y_t = B_t \mathrm{d}t + t \mathrm{d}B_t +  \cdot 0
    \]

    In the integral form,
    \[
        tB_t = \int_0^t B_s ~\mathrm{d}s + \int_0^t s ~\mathrm{d}B_s
    \]
    or
    \[
        \int_0^t s ~\mathrm{d}B_s = tB_t - \int_0^t B_s ~\mathrm{d}s
    \]
\end{example}

This example is generalized in the next theorem.

\begin{theorem}[Integration by parts]
    Suppose that $f(s, \omega) = f(s)$ only depends on $s$ and that $f$ is continuous and of bounded variation in $[0,t]$. Then 
    \begin{equation*}
        \int_0^t f(s) ~\mathrm{d}B_s = f(t) B_t - \int_0^t B_s ~\mathrm{d}f_s
    \end{equation*}
\end{theorem}

\section{The Multi-Dimensional Itô Formula}

\begin{definition}[$n$-dimensional Itô process]
    Let $B(t, \omega) = (B_{t_1}(t, \omega), \ldots , B_m(t, \omega))$ denote $m$-dimensional Brownian motion. If each $u_i(t, \omega)$ and $v_{ij}(t, \omega)$, for $1 \leq i \leq n$ and $1 \leq j \leq m$, satisfies the conditions given in the \hyperref[def:ito-process]{definition of Itô process}, then we can form $n$ Itô processes 
    \begin{equation*}
        \left\{\begin{aligned}
            \mathrm{d} X_1 &= u_1 \mathrm{d}t + v_{11}\mathrm{d}B_1 + \ldots + v_{1m}\mathrm{d}B_m \\
            &\vdots \\
            \mathrm{d} X_n &= u_n \mathrm{d}t + v_{n1}\mathrm{d}B_1 + \ldots + v_{nm}\mathrm{d}B_m \\ 
        \end{aligned}\right.
    \end{equation*}

    Or simply
    \[
        \mathrm{d}X(t) = u \mathrm{d}t + v \mathrm{d}B(t)
    \]
    where
    \[
        X(t) = \begin{bmatrix}
            X_1(t) \\
            \vdots \\
            X_n(t)
        \end{bmatrix}, \,
        u = \begin{bmatrix}
            u_1 \\
            \vdots \\
            u_n
        \end{bmatrix}, \,
        v = \begin{bmatrix}
            v_{11} && \cdots && v_{1m} \\
            \vdots && \ddots && \vdots \\
            v_{n1} && \cdots && v_{nm}
        \end{bmatrix}, \,
        \mathrm{d}B(t) = \begin{bmatrix}
            \mathrm{d}B_1(t) \\
            \vdots \\
            \mathrm{d}B_m(t)
        \end{bmatrix}
    \]

    The process $X(t)$ is called an \textbf{$n$-dimensional Itô process}.
\end{definition}

\begin{theorem}[General Itô Formula]
    Let
    \[
        \mathrm{d}X(t) = u \mathrm{d}t + v \mathrm{d}B(t)
    \]
    be an $n$-dimensional Itô process. And let $g(t,x) = (g_1(t,x), \ldots, g_p(t,x))$ be a $\mathfrak{C}^2$ map from $[0, \infty) \times \textbf{R}^n$ into $\textbf{R}^p$. Then the process 
    \[
        Y(t, \omega) = g(t, X(t))
    \]
    is an Itô process, whose $k$-th component is given by 
    \[
        \mathrm{d}Y_k = \frac{\partial g_k}{\partial t}(t, X) \mathrm{d}t + \sum_i \frac{\partial g_k}{\partial x_i}(t, X) \mathrm{d}X_i + \frac{1}{2} \sum_{i,j} \frac{\partial^2 g_k}{\partial x_i \partial x_j} (t,X) \mathrm{d}X_i \mathrm{d}X_j
    \]
    where $\mathrm{d}B_i \mathrm{d}B_j = \delta_{ij}\mathrm{d}t$, $\mathrm{d}B_i \mathrm{d}t = \mathrm{d}t \mathrm{d}B_i = 0$.
\end{theorem}

\section{The Martingale Representation Theorem}

Let $B(t)$ be an $n$-dimensional Brownian motion. We saw that if $v \in \mathfrak{V}^n$, then the Itô integral
\begin{equation*}
    X_t = X_0 + \int_0^t v(s, \omega) ~\mathrm{d}B(s), \, t \geq 0
\end{equation*}
is always a martingale w.r.t. filtration $\mathfrak{F}_t^n$ and the probability measure $\mathbb{P}$.

In this section, we show that the converse is also true: Any $\mathfrak{F}_t^n$-martingale (w.r.t to $\mathbb{P}$) can be represented as an Itô integral. This result is called the martingale representation theorem, and is important in many applications, such as mathematical finance.

\begin{lemma}
    Fix $T > 0$. The set of random variables 
    \[
        \{ \varphi(B_{t_1}, \ldots, B_{t_n}) \}
    \]
    where $t_i \in [0, T]$, $\varphi \in C_0^\infty(\textbf{R}^n)$, and $n = 1, 2, \ldots$ is dense in $L^2(\mathfrak{F}_T, \mathbb{P})$.
\end{lemma}

\begin{lemma}
    The linear span of random variables of the type
    \begin{equation*}
        \exp \left\{ \int_0^T h(t) ~\mathrm{d}B_t(\omega) - \frac{1}{2} \int_0^T H^2(t) ~\mathrm{d}t \right\}
    \end{equation*}
    where $h \in L^2[0,T]$ is dense in $L^2(\mathfrak{F}_T, \mathbb{P})$.
\end{lemma}

Suppose that $B(t)$ is $n$-dimensional. If $v(s, \omega) \in \mathfrak{V}^n(0,T)$, then the random variable
\begin{equation*}
    V(\omega) := \int_0^T v(t, \omega) ~\mathrm{d}B(t)
\end{equation*}
is $\mathfrak{F}_t^n$-measurable and by the Itô isometry
\begin{equation*}
    \mathbb{E}[V^2] = \int_0^T \mathbb{E}[v^2(t, \cdot)] ~\mathrm{d}t < \infty
\end{equation*}
so $V \in L^2(\mathfrak{F}_T, \mathbb{P})$.

\begin{theorem}[Itô Representation Theorem]
    Let $F \in L^2(\mathfrak{F}_T, \mathbb{P})$. Then there exists a unique stochastic process $f(t, \omega) \in \mathfrak{V}^n(0, T)$ such that
    \begin{equation*}
        F(\omega) = \mathbb{E}[F] + \int_0^T f(t, \omega) ~\mathrm{d}B(t)
    \end{equation*}
\end{theorem}

\begin{theorem}[The Martingale Representation Theorem]
    Let $B(t)$ be an $n$-dimensional Brownian motion. Suppose $M_t$ is an $\mathfrak{F}_t^{(n)}$-martingale w.r.t. $\mathbb{P}$ and that $M_t \in L^2(\mathbb{P})$ for all $t \geq 0$.

    Then there exists a unique stochastic process $g(s, \omega)$ such that $g \in \mathfrak{V}^{(n)}(0, t)$ for all $t \geq 0$ and 
    \begin{equation*}
        M_t(\omega) = \mathbb{E}[M_0] + \int_0^t g(s, \omega) ~\mathrm{d}B(s)
    \end{equation*}
    almost surely, for all $t \geq 0$.
\end{theorem}