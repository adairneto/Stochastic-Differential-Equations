\usepackage{pgfplots}
\pgfplotsset{width=7cm,compat=1.8}
\usepackage{pgfplotstable}
% Create a function for generating inverse normally distributed numbers using the Box–Muller transform
\pgfmathdeclarefunction{invgauss}{2}{%
  \pgfmathparse{sqrt(-2*ln(#1))*cos(deg(2*pi*#2))}%
}
\makeatletter
\pgfplotsset{
    table/.cd,
    brownian motion/.style={
        create on use/brown/.style={
            create col/expr accum={
                (\coordindex>0)*(
                    max(
                        min(
                            invgauss(rnd,rnd)*0.1+\pgfmathaccuma,
                            \pgfplots@brownian@max
                        ),
                        \pgfplots@brownian@min
                    )
                ) + (\coordindex<1)*\pgfplots@brownian@start
            }{\pgfplots@brownian@start}
        },
        y=brown, x expr={\coordindex},
        brownian motion/.cd,
        #1,
        /.cd
    },
    brownian motion/.cd,
            min/.store in=\pgfplots@brownian@min,
        min=-inf,
            max/.store in=\pgfplots@brownian@max,
            max=inf,
            start/.store in=\pgfplots@brownian@start,
        start=0
}
\makeatother

% Initialise an empty table with a certain number of rows
\pgfplotstablenew{201}\loadedtable % How many steps?
\pgfplotsset{
        no markers,
        xmin=0,
        enlarge x limits=false,
        scaled y ticks=false,
        ymin=-1, ymax=1
}
\tikzset{line join=bevel} 

% NEW COMMANDS
\newcommand\restr[2]{{% we make the whole thing an ordinary symbol
  \left.\kern-\nulldelimiterspace % automatically resize the bar with \right
  #1 % the function
  \vphantom{\big|} % pretend it's a little taller at normal size
  \right|_{#2} % this is the delimiter
  }}
  
\newcommand\crule[3][black]{\textcolor{#1}{\rule{#2}{#3}}}